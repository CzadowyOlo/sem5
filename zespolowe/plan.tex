\documentclass[12pt]{article}
\usepackage{amsmath}
\usepackage{graphicx}
\usepackage{hyperref}
\usepackage[utf8]{inputenc}
\usepackage[T1]{fontenc}
\usepackage[polish]{babel}

\author{MARIK 2, 3, 8}
\title{PLAN}
\date{02.11.2022}

\begin{document}

\maketitle

\tableofcontents

\section{Home Page}
\begin{enumerate}
    \item search bar do wyszukiwania istniejących działek H
    \item guzik do dodania nowej działki - przenosi na nowy ekran H
    \item lista istniejących działek z której mozna zakliknąć wybraną 
    dziąłkę
    i przenosi nas to na nowy ekran działki H
    \item przypomnienia (guzik) przenosi na ekran przypomnień L
    \item opcja wyróżnienia działek na liście (ulubione) i filter tylko na ulubione L

\end{enumerate}
\newpage
\section{Dodawanie nowej działki}
\begin{enumerate}
    \item search bar lokalizacji z rzutowaniem mapy rejonu
    \item menu rozwijane z narzędziami
    \begin{itemize}
        \item kursor - złap mapę i przesuń
        \item linijka
        \item swobodny rysunek
        \item zapisz
    \end{itemize}
    \item guzik: importuj zaznaczony teren do listy dzialek
\end{enumerate}



\end{document}

\newpage
\section{Okno edycji obiektu}
Funkcje dostępne dla użytkownika po kliknięciu okna edycji obiektu obiektu.
\begin{table}[H]
    \begin{center}
    \label{tab:table}
    \begin{tabularx}{1.1\textwidth} { 
    >{\raggedright\arraybackslash}X 
    | >{\raggedright\arraybackslash}X 
    | >{\raggedleft\arraybackslash}X}
    \textbf{Funkcja} & \textbf{Opis} & \textbf{Priorytet}\\
    \hline
    Przypomnienia&Okienko z przypomnieniami dla danego obiektu&M\\
    \hline
    Ilość/Liczba&Zmiana liczby obiektów, np liczba krzaków pomidora na grządce&H\\
    \hline
    Edycja parametrów&Pozwala na zmaine parametrów obiektu&H\\
    \hline
    Tagi&Pozwala dawać obiektom tagi&M\\
    \hline
    Notatki&Okienko pozawalające na dodawanie opisu i notatek do obiektu&M\\
    \hline
    Stwórz szablon&Okienko pozawalające na tworzenie szablonu&M\\
    \hline
    Usuń&Usuwa obiekt&H\\
    \hline
    \end{tabularx}
    \end{center}
    \end{table}
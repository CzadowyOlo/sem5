\documentclass[12pt]{article}
\usepackage{amsmath}
\usepackage{graphicx}
\usepackage{hyperref}
\usepackage[utf8]{inputenc}
\usepackage[T1]{fontenc}
\usepackage[polish]{babel}

\author{Aleksander Głowacki}
\title{Sprawozdanie 1}
\date{20.10.2022}

\begin{document}

\maketitle

\tableofcontents

\section{Zadanie 1.}


\subsection{Opis problemu}
Zadanie dotyczy wyznaczania szczególnych wartości liczbowych:
\begin{enumerate}
    \item epsilon maszynowy - 
    najmniejsza liczbę macheps > 0 taka, że: \\
     $f l$(1.0+macheps) > 1.0 i $f l$(1.0+macheps) = 1+macheps
    \item eta - liczba maszynowa eta to najmniejsza wartość większa od 0.0
    \item MAX - największa wartość liczbową jaką możemy określić w danej arytmetyce
  \end{enumerate}
\subsection{Sposób rozwiązania}
Sprawdzam w pętli czy zachodzi określony warunek z szukaną liczbą\\
względem liczby $one(typ)$ lub $zero(typ)$. 
(te funkcje zwracają mi wartość jedynki i zera \\
w danym sytemie.) \\Gdy warunek zostanie złamany to znajduję szukaną wartość.
\begin{enumerate}
    \item $\varepsilon$ - x \\ $cond:$ $one(type) + x/2 \neq one(type)$ 
    \\ zaczynam z $x = one(type)$ a  w pętli dzielę $x/2$
    \item $\eta$ - x \\ $cond:$ $x/2 \neq zero(type)$ \\ 
    zaczynam z $x = one(type)$ a  w pętli dzielę $x/2$ 
    \item $\text{MAX}$ - x \\ w pierwszej pętli zaczynamy od 
    $x = one(type)$. W środku pętli mnożymy x razy 2 i zatrzymujemy się, gdy\\ $x*2 = inf$\\
    Następnie do naszego x dodajemy uzupełnienie o postaci $dodatek = x/2$.\\
    Zatrzymujemy się gdy $x + dodatek/2 = inf$
\end{enumerate}
\newpage
\subsection{Wyniki}
    \begin{table}[h]
        \caption{macheps}
        \label{epsilon}
        %\centering
        \begin{tabular}{|l|l|l|l|}
            \hline 
            \textbf{typ danych} & \textbf{moja funkcja} & \textbf{funkcja biblioteczna} & \textbf{float.h}\\
            \hline
            \textbf{Float16} & 0.000977 & 0.000977 & brak\\
            \hline
            \textbf{Float32} & 1.1920929e-7 & 1.1920929e-7 & 1.192093e-7\\
            \hline
            \textbf{Float64} & 2.220446049250313e-16 & 2.220446049250313e-16 & 2.220446e-16\\
            \hline
        \end{tabular} 
    \end{table}
    \begin{table}[h]
        \caption{eta}
        \label{eta}
        %\centering
        \begin{tabular}{|l|l|l|}
            \hline 
            \textbf{typ danych} & \textbf{moja funkcja} & \textbf{funkcja biblioteczna}\\
            \hline
            \textbf{Float16} & 6.0e-8 & 6.0e-8\\
            \hline
            \textbf{Float32} & 1.0e-45 & 1.0e-45\\
            \hline
            \textbf{Float64} & 5.0e-324 & 5.0e-324 \\
            \hline
        \end{tabular} 
    \end{table}

    \begin{table}[h]
        \caption{MAX}
        \label{max}
        %\centering
        \begin{tabular}{|l|l|l|}
            \hline 
            \textbf{typ danych} & \textbf{moja funkcja} & \textbf{funkcja biblioteczna}\\
            \hline
            \textbf{Float16} & 6.55e4 & 6.55e4\\
            \hline
            \textbf{Float32} & 3.4028235e38 & 3.4028235e38\\
            \hline
            \textbf{Float64} & 1.7976931348623157e308 & 1.7976931348623157e308 \\
            \hline
        \end{tabular} 
    \end{table}

\subsection{Wnioski}
\begin{enumerate}
    \item Nie można zaprezentować wszystkich liczb rzeczywistych na komputerze\\
    W poszczególnych typach danych mamy określoną gęstość liczb oraz zakres.
    \item Porównując eta z MINsub i floatmin(type) z MINnor zgadza się tylko rząd wielkości
    \item $macheps = 2* \text{precyzja arytmetyki}$
\end{enumerate}
\section{Zadanie 2.}

\subsection{Opis problemu}

Wyznaczenie epsilona maszynowego używając formuły:\\\centerline{$3(4/3-1)-1$}

\subsection{Sposób rozwiązania}
Przepisanie formuły pamiętając o właściwym podstawieniu wartości 1.0
\subsection{Wyniki}
\begin{table}[h]
    \caption{macheps}
    \label{macheps}
    %\centering
    \begin{tabular}{|l|l|l|}
        \hline 
        \textbf{typ danych} & \textbf{moja funkcja} & \textbf{funkcja biblioteczna}\\
        \hline
        \textbf{Float16} & -0.000977 & 0.000977\\
        \hline
        \textbf{Float32} & 1.1920929e-7  & 1.1920929e-7\\
        \hline
        \textbf{Float64} & -2.220446049250313e-16 & 2.220446049250313e-16 \\
        \hline
    \end{tabular} 
\end{table}
\subsection{Wnioski}
\begin{enumerate}
    \item Otrzymane przeze mnie wartości mają zgodny moduł z wartościami z funkcji $eps(type)$
\end{enumerate}
\section{Zadanie 3.}

\subsection{Opis problemu}
Zbadać gęstość przedziałów liczbowych:
\begin{enumerate}
    \item $[\frac{1}{2}, 1)$
    \item $[1, 2)$
    \item $[2, 4)$
\end{enumerate}
\subsection{Sposób rozwiązania}
\begin{enumerate}
    \item Iterujemy się po przedziale $[1, 2)$ z krokiem $\delta = {2^{-52}}$\\
    i sprawdzamy czy otrzymaliśmy $nextfloat(liczba)$
    \item Nie znamy kroku więc jest trudniej.\\
    Wiemy, że przesunięcie w kodowaniu eksponenty wynosi $1023_{10}$\\
    Znamy własności systemu dwójkowego. Eksponenta zmienia się nam
    w kolejnych potęgach liczby 2. Zatem przedziały o takiej samej eksponencie
    przy dążeniu do $\infty$ będą sie nam rozciągać. \\Jednak ilość bitów w eksponencie i mantysie\\
    jest stała. \\Co skłania nas do przypuszczenia, że ilość liczb w danym przedzialale\\
    pomiędzy potęgami 2 jest stała.
    Funkcją $bitsring$ sprawdzamy czy eksponenta krańcowych liczb z przedziału jest identyczna.\\
    Zwracamy wartość ${2^{e - 1023 - 52}} $, gdzie $e$ jest eksponentą, 1023 to przesunięcie,
    a 52 to liczba bitów mantysy

\end{enumerate}

\subsection{Wyniki}
\begin{table}[h]
    \caption{macheps}
    \label{macheps1}
    %\centering
    \begin{tabular}{|c|c|c|}
        \hline 
        \textbf{$[\frac{1}{2}, 1)$} & \textbf{$[1, 2)$} & \textbf{$[2, 4)$}\\
        \hline
        \ 1.1102230246251565e-16 & 2.220446049250313e-16 & 4.440892098500626e-16\\
        \hline
    \end{tabular} 
\end{table}
\subsection{Wnioski}
\begin{enumerate}
    \item Im bliżej zera, tym gęstsze mamy liczby w systemie IEE754 podwójnej precyzji
    \item kroki pomiędzy liczbami rzeczywistymi wynoszą kolejno:
    \begin{enumerate}
        \item ${2^{-53}}$
        \item ${2^{-52}}$
        \item ${2^{-51}}$
    \end{enumerate}
    \item Przeskoki w gęstościach pokrywają się z przeskokami w potęgach dwójki
\end{enumerate}
\section{Zadanie 4.}

\subsection{Opis problemu}
Znaleźć liczbę $x \in (1, 2)$ taką, że $x * \frac{1}{x} \neq x $
\subsection{Sposób rozwiązania}
sprawdzać ten warunek w pętli od $nextfloat(one(Float64))$ aktualizując\\
$x := nextfloat(x)$
\subsection{Wyniki}
$x = 1.000000057228997$
\subsection{Wnioski}
W systemie Float64 dzielenie przez liczbę nie jest tożsame z mnożeniem przez odwrotność.

\section{Zadanie 5.}

\subsection{Opis problemu}
Różne sposoby obliczenia iloczynu skalarnego zadanych wektorów.
\subsection{Sposób rozwiązania}
Przepisanie algorytmu z treści zadania do Julii.
\newpage
\subsection{Wyniki}
Wartość dokładna: $-1.00657107000000 * {10^{-11}}$
\begin{table}[h]
    \caption{wyniki we Float32}
    \label{vectors}
    \centering
    \begin{tabular}{|l|c|}
        \hline 
        \textbf{nr algorytmu} & \textbf{wynik} \\
        \hline
        \ 1 & -0.3472038161853561\\
        \hline
        \ 2 & -0.3472038162872195\\
        \hline
        \ 3 & -0.5\\
        \hline
        \ 4 & -0.5\\
        \hline
    \end{tabular} 
\end{table}
\begin{table}[h]
    \caption{wyniki we Float64}
    \label{vectors2}
    \centering
    \begin{tabular}{|l|c|}
        \hline 
        \textbf{nr algorytmu} & \textbf{wynik} \\
        \hline
        \ 1 & 1.0251881368296672e-10\\
        \hline
        \ 2 & -1.5643308870494366e-10\\
        \hline
        \ 3 & 0.0\\
        \hline
        \ 4 & 0.0\\
        \hline
    \end{tabular} 
\end{table}
\subsection{Wnioski}
Kolejność wykonywania działań mocno zmienia końcowy wynik.\\
Wynika to ze względu na utratę precyzji na różnych etapach obliczeń.\\
Czasem sie straci więcej, czasem mniej. \\
Tak czy siak każdy wynik ma niewiele wspólnego
z rzeczywistością.
\section{Zadanie 6.}

\subsection{Opis problemu}
Sprawdzić, czy dwie funkcje inaczej zapisane, ale matematycznie równoważne\\
 zwracają te same wartości dla pewnych danych.
\begin{center}
   $f(x) = \sqrt[2]{{x^{2}} + 1} - 1$\\
\end{center}
\begin{center}
    $g(x) = \frac{{x^{2}}}{\sqrt[2]{{x^{2}} + 1} + 1}$
\end{center}
\subsection{Sposób rozwiązania}
Przepisać bezpośrednio powyższe funkcje \\
i przetestować je dla kolejnych wartości o postaci:\\
$ x = {8^{-k}}$
\subsection{Wyniki}
\begin{table}[h]
    \caption{wartości funkcji}
    \label{wartościfg}
    \centering
    \begin{tabular}{|l|c|c|}
        \hline 
        \textbf{$k$} & textbf{$f({8^{-k}})$} & \textbf{$g({8^{-k}})$} \\
        \hline
        0 & 0.41421356237309515 & 0.4142135623730951\\
        \hline
        1 & 0.0077822185373186414 & 0.0077822185373187065\\
        \hline
        2 & 0.00012206286282867573 & 0.00012206286282875901\\
        \hline
        3 & 1.9073468138230965e-6 & 1.907346813826566e-6\\
        \hline
        4 & 2.9802321943606103e-8 & 2.9802321943606116e-8\\
        \hline
        5 & 4.656612873077393e-10 & 4.6566128719931904e-10\\
        \hline
        6 & 7.275957614183426e-12 & 7.275957614156956e-12\\
        \hline
        7 & 1.1368683772161603e-13 & 1.1368683772160957e-13\\
        \hline
        8 & 1.7763568394002505e-15 & 1.7763568394002489e-15\\
        \hline
        9 & 0.0 & 2.7755575615628914e-17\\
        \hline
        10 & 0.0 & 4.336808689942018e-19\\
        \hline
        11 & 0.0 & 6.776263578034403e-21\\
        \hline
        12 & 0.0 & 1.0587911840678754e-22\\
        \hline
        13 & 0.0 & 1.6543612251060553e-24\\
        \hline
        14 & 0.0 & 2.5849394142282115e-26\\
        \hline
        15 & 0.0 & 4.0389678347315804e-28\\
        \hline
        16 & 0.0 & 6.310887241768095e-30\\
        \hline
        17 & 0.0 & 9.860761315262648e-32\\
        \hline
        18 & 0.0 & 1.5407439555097887e-33\\
        \hline
        19 & 0.0 & 2.407412430484045e-35\\
        \hline
        20 & 0.0 & 3.76158192263132e-37\\
        \hline
        21 & 0.0 & 5.877471754111438e-39\\
        \hline
        22 & 0.0 & 9.183549615799121e-41\\
        \hline
    \end{tabular} 
\end{table}
\subsection{Wnioski}
    \begin{enumerate}
        \item Funkcja f i g dają różne wyniki, przy czym g jest dokładniejsza
        \item Funkcja f w pewnym momencie traci całą precyzję i spada do zera
        \item Funkcja f radzi sobie gorzej ze względu na odejmowanie 1 od pierwiastka\\
        gdzie traci precyzję
        \item Należy tak dobierać kolejność działań, żeby liczba cyfr znaczących nie różniła się zbytnio
    \end{enumerate}
\section{Zadanie 7.}

\subsection{Opis problemu}
Przybliżenie wartości pochodnej funkcji $f(x)$ w punkcie $x$
\subsection{Sposób rozwiązania}
Podstawienie wszystkiego do zadanego wzoru, obliczenie pochodnej na papierze.
\newpage
\subsection{Wyniki}
Wartość dokladna: $f'(x) = 0.11694228168853815$
\begin{table}[h]
    \caption{wartości funkcji}
    \label{przyblizenia}
    \centering
    \begin{tabular}{|c|c|c|c|}
        \hline 
        $n$ & przybliżona pochodna & błąd & 1+h \\ \hline
0 & 2.0179892252685967 & 1.9010469435800585 & 2.0 \\ \hline
1 & 1.8704413979316472 & 1.753499116243109 & 1.5 \\ \hline
2 & 1.1077870952342974 & 0.9908448135457593 & 1.25 \\ \hline
3 & 0.6232412792975817 & 0.5062989976090435 & 1.125 \\ \hline
4 & 0.3704000662035192 & 0.253457784514981 & 1.0625 \\ \hline
5 & 0.24344307439754687 & 0.1265007927090087 & 1.03125 \\ \hline
9 & 0.1248236929407085 & 0.007881411252170345 & 1.001953125 \\ \hline
10 & 0.12088247681106168 & 0.0039401951225235265 & 1.0009765625 \\ \hline
15 & 0.11706539714577957 & 0.00012311545724141837 & 1.000030517578125 \\ \hline
16 & 0.11700383928837255 & 6.155759983439424e-5 & 1.0000152587890625 \\ \hline
18 & 0.11695767106721178 & 1.5389378673624776e-5 & 1.0000038146972656 \\ \hline
27 & 0.11694231629371643 & 3.460517827846843e-8 & 1.0000000074505806 \\ \hline
28 & 0.11694228649139404 & 4.802855890773117e-9 & 1.0000000037252903 \\ \hline
29 & 0.11694222688674927 & 5.480178888461751e-8 & 1.0000000018626451 \\ \hline
30 & 0.11694216728210449 & 1.1440643366000813e-7 & 1.0000000009313226 \\ \hline
35 & 0.11693954467773438 & 2.7370108037771956e-6 & 1.0000000000291038 \\ \hline
37 & 0.1169281005859375 & 1.4181102600652196e-5 & 1.000000000007276 \\ \hline
38 & 0.116943359375 & 1.0776864618478044e-6 & 1.000000000003638 \\ \hline
39 & 0.11688232421875 & 5.9957469788152196e-5 & 1.000000000001819 \\ \hline
41 & 0.116943359375 & 1.0776864618478044e-6 & 1.0000000000004547 \\ \hline
43 & 0.1162109375 & 0.0007313441885381522 & 1.0000000000001137 \\ \hline
50 & 0.0 & 0.11694228168853815 & 1.0000000000000009 \\ \hline
51 & 0.0 & 0.11694228168853815 & 1.0000000000000004 \\ \hline
52 & -0.5 & 0.6169422816885382 & 1.0000000000000002 \\ \hline
53 & 0.0 & 0.11694228168853815 & 1.0 \\ \hline
54 & 0.0 & 0.11694228168853815 & 1.0 \\ \hline
    \end{tabular} 
\end{table}
\newpage
\subsection{Wnioski}
\begin{enumerate}
    \item Największa precyzja jest dla n = 28
    \item Błąd rośnie symetrycznie względem środka (n=28) 
    \item Badając pochodną musimy unikać wartości zbliżonych do 0 w trakcie obliczeń,\\
    ponieważ tracimy wtedy precyzję. Dla małego n mamy niedokładny wynik z powodu słabego parametru\\
    ale zwiększanie go do oporu nic nam nie da, bo tracimy precyzję gdy zbliża sie on do 0
\end{enumerate}

\end{document}

\documentclass{article}

\usepackage{polski}

\usepackage{graphicx}
\usepackage{amsmath}
\usepackage{amsfonts}
\usepackage{listings}
\usepackage{longtable}
\usepackage[a4paper, total={6in, 8in}]{geometry}



\author{Paweł Wilkosz}
\title{Obliczenia Naukowe - sprawozdanie nr 1}

\begin{document}
\maketitle

\section{Parametry arytmetyki zmiennopozycyjnej}

\subsection{Epsilon maszynowy}

Epsilon maszynowy to jeden z najwazniejszych parametrów arytmetyki zmiennopozycyjnej.
Można go zdefiniować jako najmniejsze takie $x$, że $1+x > 1$.
W poniższej tabeli przedstawione są wartości epsilona maszynowego zwracane przez funkcję \texttt{eps} ze standardowej biblioteki języka Julia.

\begin{center}
  \begin{tabular}{| c | c |}
    \hline
    Typ & \texttt{eps} \\
    \hline
    Float16 & 0.000977 \\
    Float32 & 1.1920929e-7 \\
    Float64 & 2.220446049250313e-16 \\
    \hline
  \end{tabular}
\end{center}

Parametr ten można również obliczyć iteracyjnie, za pomocą poniższej procedury.
\begin{lstlisting}
function itereps(type)
  current_num = one(type)
  while (one(type) + current_num / 2) > one(type)
    current_num /= 2
  end
  return current_num
end
\end{lstlisting}

Uruchomienie powyższej funkcji daje wyniki zgodne z rezultatami funkcji \texttt{eps}.
\begin{center}
  \begin{tabular}{| c | c |}
    \hline
    Typ & \texttt{itereps} \\
    \hline
    Float16 & 0.000977 \\
    Float32 & 1.1920929e-7 \\
    Float64 & 2.220446049250313e-16 \\
    \hline
  \end{tabular}
\end{center}

\subsection{Liczba Eta}

Liczba Eta, równa wspomnianej na wykładzie liczby $MIN_{sub}$ to najmniejsza (co do wartości bezwzględnej) nieznormalizowana liczba reprezentowalna w danej arytmetyce.

\begin{center}
  \begin{tabular}{| c | c |}
    \hline
    Typ & \texttt{itereta} \\
    \hline
    Float16 & 6.0e-8 \\
    Float32 & 1.0e-45 \\
    Float64 & 5.0e-324 \\
    \hline
  \end{tabular}
\end{center}

\subsection{Float min i Float max}

Liczby \texttt{floatmin} i \texttt{floatmax} opisują odpowiednio, minimalne i maksymalne znormalizowane liczby dodatnie możliwe do przedstawienia w danej arytmetyce.
Poniższa tabela przedstawia wartości zwracane przez funkcję \texttt{floatmin} dla wszystkich typów zmiennoprzecinkowych w języku Julia.

\begin{center}
  \begin{tabular}{| c | c |}
    \hline
    Typ & \texttt{floatmin} \\
    \hline
    Float16 & 6.104e-5 \\
    Float32 & 1.1754944e-38 \\
    Float64 & 2.2250738585072014e-308 \\
    \hline
  \end{tabular}
\end{center}

Liczbę \texttt{floatmax} możemy obliczyć zarówno przy pomocy wbudowanej funkcji języka Julia jak i iteracyjnie, przy pomocy poniższej metody.

\begin{lstlisting}
function itermax(type)
 current_num = prevfloat(one(type))
 while !isinf(current_num * 2)
  current_num *= 2
 end
 return current_num
end
\end{lstlisting}

Obliczona liczba jest efektem kolejnego mnożenie przez dwa liczby \texttt{prevfloat(one(type))}.
Wybór początkowej liczby podyktowany jest tym, że ma ona maksymalną mantysę, wystarczy więc zwiększyć cechę - mnożąc przez dwa.

W poniższej tabeli przedstawione są wartości zwracane przez funkcję \texttt{itermax} porównane z wynikami funkcji \texttt{floatmax}.

\begin{center}
  \begin{tabular}{| c | c | c |}
    \hline
    Typ & \texttt{itermax} & \texttt{floatmax} \\
    \hline
    Float16 & 6.55e4 & 6.55e4 \\
    Float32 & 3.4028235e38 & 3.4028235e38 \\
    Float64 & 1.7976931348623157e308 & 1.7976931348623157e308 \\
    \hline
  \end{tabular}
\end{center}

Równość wyników \texttt{itermax} i \texttt{floatmax} dowodzi poprawności wyżej opisanego algorytmu.

\subsection{Podsumowanie}

Arytmetyka IEEE-754 niesie za sobą wiele ograniczeń.
Możemy przekonać się o nich eksperymentalnie, używając wbudowanych funkcji języka Julia, bądź samodzielnie sprawdzając implementację.
Zajrzywszy do pliku \texttt{float.h} kompilatora \texttt{gcc} przekonamy się, że zawarte tam wartości są zgodne z tymi, które wynikają z powyższych eksperymentów.


\section{Epsilon maszynowy metodą Kahana}

Do obliczenia epsilona maszynowego posłużyć może również wyrażenie:
$$3(\frac{4}{3} - 1) - 1$$
Implementując je w języku Julia i uruchamiając kolejno dla odpowiednich arytmetyk otrzymujemy:
\begin{center}
  \begin{tabular}{| c | c |}
    \hline
    Typ & Obliczony epsilon \\
    \hline
    Float16 & -0.000977 \\
    Float32 & 1.1920929e-7 \\
    Float64 & -2.220446049250313e-16 \\
    \hline
  \end{tabular}
\end{center}

Jak łatwo zaobserwować, metoda ta zwraca idealną wartość epsilona maszynowego z dokładnością co do wartości bezwzględnej.
Dla typów \texttt{Float16} i \texttt{Float64} wartości są ujemne.

Takie wyniki spowodowane są zaokrągleniem liczby $\frac{4}{3}$ w pamięci komputera.
Liczba ta w systemie binarnym ma nieskończone rozwinięcie więc w zależności od parzystości długości mantysy, faktyczna wartość w pamięci jest większa bądź mniejsza o \texttt{macheps} od porządanej przez nas wartości.

\section{Rozmieszczenie liczb zmiennoprzecinkowych}

Celem zadania jest sprawdzenie czy liczby z zakresu $[1,2]$ są rozmieszczone rónomiernie z krokiem $\delta = 2^{-52}$.

W celu weryfikacji tego faktu, można użyć funkcji \texttt{bitstring}.
Jeżeli postawiona teza jest prawdziwa, dodanie do kolejnych liczb z zakresu liczby $\delta$ powinno powodować zwiększenie mantysy o 1.

\begin{lstlisting}
bitstring(1+delta) 
0011111111110000000000000000000000000000000000000000000000000001
bitstring(1+2*delta)
0011111111110000000000000000000000000000000000000000000000000011
bitstring(1+3*delta)
0011111111110000000000000000000000000000000000000000000000000100
\end{lstlisting}

Wykonanie obliczeń dla pierwszych trzech liczb z zakresu potwierdza tezę.

Własność tę można uogólnić dla wszystkich zakresów liczb ze stałą cechą, to jest postaci $[2^k, 2^{k+1}]$ gdzie $k \in \mathbb{Z}$.
Jako że liczba możliwych mantys jest stała, w każdym z takich zakresów znajduje się tyle samo liczb.
Można przetestować tę własność na zakresach $[\frac{1}{2}, 1]$ oraz $[2, 4]$.
W związku z tym, że zakresy te są odpowiednio dwa razy mniejsze lub dwa razy większe od początkowo badanego, krok pomiędzy poszczególnymi liczbami również musi ulec zmianie.

Dla liczb z zakresu $[2,4]$ krok wynosi $\delta=2^{-51}$
\begin{lstlisting}
bitstring(2+delta) 
0100000000000000000000000000000000000000000000000000000000000001
bitstring(2+2*delta) 
0100000000000000000000000000000000000000000000000000000000000011
bitstring(2+3*delta) 
0100000000000000000000000000000000000000000000000000000000000100
\end{lstlisting}
a dla liczb z zakresu $[\frac{1}{2}, 1]$ krok wynosi $\delta=2^{-53}$
\begin{lstlisting}
bitstring(0.5+delta) 
0011111111100000000000000000000000000000000000000000000000000001
bitstring(0.5+2*delta) 
0011111111100000000000000000000000000000000000000000000000000011
bitstring(0.5+3*delta) 
0011111111100000000000000000000000000000000000000000000000000100
\end{lstlisting}

Z powyższych doświadczeń wynika, iż różnica pomiędzy kolejnymi liczbami zmiennoprzecinkowymi rośnie wraz z oddalaniem się od zera.
Co za tym idzie, im większa cecha, tym mniejsza bezwzględna dokładność obliczeń.

\section{Odwracalność liczb}

Zgodnie ze standardową matematyką, wyrażenie $x \cdot \frac{1}{x} = 1$ jest zawsze prawdziwe dla $x \neq 0$.
Niestety w arytmetyka zmiennoprzecinkowa sprawia, że równanie to nie jest prawdziwe dla wszystkich liczb.

Aby znaleźć najmniejszą liczbę niespełniającą tego warunku w zakresie $[1.0, 2.0]$ możemy użyć poniższej procedury.

\begin{lstlisting}
function find_irreversible()
  number = one(Float64)
  while isreversible(number) && number < 2.0
    number = nextfloat(number)
  end
  return number
end
\end{lstlisting}

Wynikiem jej działania jest liczba \texttt{1.000000057228997}.

Istnienie takiej liczby jest spowodowane ograniczonym systemem reprezentacji liczb w pamięci.
Wiele liczb rzeczywistych przedstawialnych w systemie dziesiętnym ze skończonym rozwinięciem, w systemie binarnym posiada rozwinięcie nieskończone.
Powoduje to duże utraty dokładności podczas dzielenia niecałkowitoliczbowego.

\section{Iloczyn skalarny dwóch wektorów}

Zadanie bada problem liczenia iloczynu skalarnego dwóch wektorów.
Porównujemy cztery algorytmy służące do tego celu.

\begin{enumerate}
  \item suma iloczynów kolejnych współrzędnych wykonana od początku tablicy do końca \texttt{fronttoback}
  \item suma iloczynów kolejnych współrzędnych wykonana w odwrotnej kolejności \texttt{backtofront}
  \item suma iloczynów kolejnych współrzednych wykonana w kolejności od wyników najdalszych zeru do najbliższych, z podziałem na dodatnie i ujemne \texttt{bigtosmall}
  \item suma iloczynów kolejnych współrzednych wykonana w kolejności od wyników najbliższych zeru do najdalszych, z podziałem na dodatnie i ujemne \texttt{smalltobig}
\end{enumerate}

W poniższej tabeli przedstawiono wyniki eksperymentu dla danych 

$$
x = [2.718281828, -3.141592654, 1.414213562, 0.5772156649, 0.3010299957]
$$
$$
y = [1486.2497,878366.9879,-22.37492,4773714.647,0.000185049]
$$

\begin{center}
  \begin{tabular}{| c | c | c |}
    \hline
    Float32 & \texttt{fronttoback} & -0.2499443 \\
    & \texttt{backtofront} & -0.2043457 \\
    & \texttt{bigtosmall} & -0.25 \\
    & \texttt{smalltobig} & -0.25 \\
    \hline
    Float64 & \texttt{frontoback} & 1.0251881368296672e-10 \\
    & \texttt{backtofront} & -1.5643308870494366e-10 \\
    & \texttt{bigtosmall} & 0.0 \\
    & \texttt{smalltobig} & 0.0 \\
    \hline
  \end{tabular}
\end{center}

Porównując te wyniki z poprawnym wynikiem wynoszącym $1.02519*10^{-10}$, najlepsze wyniki zagwarantowała arytmetyka \texttt{Float64} wraz z metodą \texttt{fronttoback}.
Metoda \texttt{backtofront} zwróciła wyniki poprawnego rzędu jednak z istotnym błędem i przeciwnym znakiem.
Metody \texttt{bigtosmall} i \texttt{smalltobig} zwróciły niepoprawne wyniki.
W arytmetyce \texttt{Float32} wszystkie wyniki są niepoprawne ze względu na utratę dokładności przy zrównywaniu cech przy dodawaniu liczb różnego rzędu

Eksperyment ten ujawnia niecodzienną własność dodawania w arytmetyce IEEE-754, mianowicie brak przemienności.


\section{Obliczanie wartości przekształconych funkcji}

Zadanie polegało na obliczeniu wartości dwóch równoważnych funkcji.
$$
f(x) = \sqrt{x^2 + 1} - 1
$$
$$
g(x) = \frac{x^2}{\sqrt{x^2 + 1} + 1}
$$

W poniższej tabeli zaprezentowane jest porównanie wyników obu funkcji dla wartości $x$ ze zbioru $\{8^{-i} : i \in [10]\}$

\begin{center}
  \begin{tabular}{| c | c | c |}
    \hline
    $x$ & $f(x)$ & $g(x)$ \\
    \hline
    $8^{-1}$ & 0.0077822185373186414 & 0.0077822185373187065 \\
    $8^{-2}$ & 0.00012206286282867573 & 0.00012206286282875901 \\
    $8^{-3}$ & 1.9073468138230965e-6 & 1.907346813826566e-6 \\
    $8^{-4}$ & 2.9802321943606103e-8 & 2.9802321943606116e-8 \\
    $8^{-5}$ & 4.656612873077393e-10 & 4.6566128719931904e-10 \\
    $8^{-6}$ & 7.275957614183426e-12 & 7.275957614156956e-12 \\
    $8^{-7}$ & 1.1368683772161603e-13 & 1.1368683772160957e-13 \\
    $8^{-8}$ & 1.7763568394002505e-15 & 1.7763568394002489e-15 \\
    $8^{-9}$ & 0.0 & 2.7755575615628914e-17 \\
    $8^{-10}$ & 0.0 & 4.336808689942018e-19 \\
    \hline
  \end{tabular}
\end{center}

Można zaobserwować, że pomimo matematycznej równości pomiędzy funkcjami, obliczenia wykonane w języku Julia zwracają różne wyniki.
Dla większych wartości, różnice są niewielkie, jednak gdy argumenty funkcji są odpowiednio bliskie $0$, wartości funkcji $f$ wynoszą $0$.
Wartości funkcji $g$ natomiast wciąż zwraca niezerowe liczby.
Na tej podstawie można wnioskować wyższą dokładność drugiej z badanych funkcji.
Może to być spowodowane małą dokładnością odejmowania bliskich sobie liczb, którego unikamy stosując funkcję $g$.

\section{Obliczanie przybliżonej wartości pochodnej}

Zadanie polegało na obliczeniu przybliżonej wartości pochodnej przy pomocy wzoru:
$$
\widetilde{f'}(x_0) = \frac{f(x_0+h) - f(x_0)}{h}
$$
gdzie przez $h$ oznaczamy dokładność obliczeń.

Badaną funkcją była $f(x) = \sin x + \cos 3x$, której pochodna dana jest wzorem $f'(x) = \cos x - 3 \sin 3x$.
Poniższa tabela przedstawia przybliżone wartości pochodnej oraz błąd obliczeń w punkcie $x_0 = 1$ dla dokładności $h \in \{2^{-i}: i \in [54]\}$.

\begin{longtable}[c]{|c|c|c|}
  \hline
  $h$ & $\widetilde{f'}(x)$ & $|\widetilde{f'}(x) - f'(x)|$ \\
  \hline
  $2^{0}$ & 2.0179892252685967 & 1.9010469435800585 \\
  $2^{-1}$ & 1.8704413979316472 & 1.753499116243109 \\
  $2^{-2}$ & 1.1077870952342974 & 0.9908448135457593 \\
  $2^{-3}$ & 0.6232412792975817 & 0.5062989976090435 \\
  $2^{-4}$ & 0.3704000662035192 & 0.253457784514981 \\
  $2^{-5}$ & 0.24344307439754687 & 0.1265007927090087 \\
  $2^{-6}$ & 0.18009756330732785 & 0.0631552816187897 \\
  $2^{-7}$ & 0.1484913953710958 & 0.03154911368255764 \\
  $2^{-8}$ & 0.1327091142805159 & 0.015766832591977753 \\
  $2^{-9}$ & 0.1248236929407085 & 0.007881411252170345 \\
  $2^{-10}$ & 0.12088247681106168 & 0.0039401951225235265 \\
  $2^{-11}$ & 0.11891225046883847 & 0.001969968780300313 \\
  $2^{-12}$ & 0.11792723373901026 & 0.0009849520504721099 \\
  $2^{-13}$ & 0.11743474961076572 & 0.0004924679222275685 \\
  $2^{-14}$ & 0.11718851362093119 & 0.0002462319323930373 \\
  $2^{-15}$ & 0.11706539714577957 & 0.00012311545724141837 \\
  $2^{-16}$ & 0.11700383928837255 & 6.155759983439424e-5 \\
  $2^{-17}$ & 0.11697306045971345 & 3.077877117529937e-5 \\
  $2^{-18}$ & 0.11695767106721178 & 1.5389378673624776e-5 \\
  $2^{-19}$ & 0.11694997636368498 & 7.694675146829866e-6 \\
  $2^{-20}$ & 0.11694612901192158 & 3.8473233834324105e-6 \\
  $2^{-21}$ & 0.1169442052487284 & 1.9235601902423127e-6 \\
  $2^{-22}$ & 0.11694324295967817 & 9.612711400208696e-7 \\
  $2^{-23}$ & 0.11694276239722967 & 4.807086915192826e-7 \\
  $2^{-24}$ & 0.11694252118468285 & 2.394961446938737e-7 \\
  $2^{-25}$ & 0.116942398250103 & 1.1656156484463054e-7 \\
  $2^{-26}$ & 0.11694233864545822 & 5.6956920069239914e-8 \\
  $2^{-27}$ & 0.11694231629371643 & 3.460517827846843e-8 \\
  $2^{-28}$ & 0.11694228649139404 & 4.802855890773117e-9 \\
  $2^{-29}$ & 0.11694222688674927 & 5.480178888461751e-8 \\
  $2^{-30}$ & 0.11694216728210449 & 1.1440643366000813e-7 \\
  $2^{-31}$ & 0.11694216728210449 & 1.1440643366000813e-7 \\
  $2^{-32}$ & 0.11694192886352539 & 3.5282501276157063e-7 \\
  $2^{-33}$ & 0.11694145202636719 & 8.296621709646956e-7 \\
  $2^{-34}$ & 0.11694145202636719 & 8.296621709646956e-7 \\
  $2^{-35}$ & 0.11693954467773438 & 2.7370108037771956e-6 \\
  $2^{-36}$ & 0.116943359375 & 1.0776864618478044e-6 \\
  $2^{-37}$ & 0.1169281005859375 & 1.4181102600652196e-5 \\
  $2^{-38}$ & 0.116943359375 & 1.0776864618478044e-6 \\
  $2^{-39}$ & 0.11688232421875 & 5.9957469788152196e-5 \\
  $2^{-40}$ & 0.1168212890625 & 0.0001209926260381522 \\
  $2^{-41}$ & 0.116943359375 & 1.0776864618478044e-6 \\
  $2^{-42}$ & 0.11669921875 & 0.0002430629385381522 \\
  $2^{-43}$ & 0.1162109375 & 0.0007313441885381522 \\
  $2^{-44}$ & 0.1171875 & 0.0002452183114618478 \\
  $2^{-45}$ & 0.11328125 & 0.003661031688538152 \\
  $2^{-46}$ & 0.109375 & 0.007567281688538152 \\
  $2^{-47}$ & 0.109375 & 0.007567281688538152 \\
  $2^{-48}$ & 0.09375 & 0.023192281688538152 \\
  $2^{-49}$ & 0.125 & 0.008057718311461848 \\
  $2^{-50}$ & 0.0 & 0.11694228168853815 \\
  $2^{-51}$ & 0.0 & 0.11694228168853815 \\
  $2^{-52}$ & -0.5 & 0.6169422816885382 \\
  $2^{-53}$ & 0.0 & 0.11694228168853815 \\
  $2^{-54}$ & 0.0 & 0.11694228168853815 \\
  \hline
\end{longtable}

Jak można zaobserwować, wraz ze spadkiem $h$ dokładność przybliżenia wartości pochodnej jest większa do pewnej warości.
Błąd obliczeń jest minimalny dla $h=2^{-28}$ a dla mniejszych wartości $h$ rośnie.
Wzrost błędu obliczeń można wytłumaczyć tym, że w arytmetyce IEEE-754 dla bardzo małych $h$, zachodzi $1+h = 1$.

\section{Podsumowanie}

Arytmetyka IEEE-754 pozwala na dokonywanie efektywne zapiwywanie liczb w pamięci komputera i dokonywanie na nich obliczeń.
Jej dokładność jest jednak ograniczona, o czym należy pamiętać wykonując obliczenia wymagającej dużej precyzji.
W szczególności na błąd jesteśmy narażeni wykonując obliczenia na liczbach różnego rzędu, to jest bardzo bliskich i bardzo dalekich od zera.

\end{document}
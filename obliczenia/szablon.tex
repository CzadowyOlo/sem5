\documentclass[12pt]{article}
\usepackage{amsmath}
\usepackage{graphicx}
\usepackage{hyperref}
\usepackage[utf8]{inputenc}
\usepackage[T1]{fontenc}
\usepackage[polish]{babel}

\author{Aleksander Głowacki}
\title{Sprawozdanie nr}
\date{dupa.dupa.2022}

\begin{document}

\maketitle

\tableofcontents

\section{Zadanie 1.}


\subsection{Opis problemu}
\subsection{Sposób rozwiązania}

\subsection{Wyniki}
    \begin{table}[h]
        \caption{blank}
        \label{epsilblankon}
        %\centering
        \begin{tabular}{|l|l|l|l|}
            \hline 
            \textbf{blank} & \textbf{blank } & \textbf{blank } & \textbf{blank}\\
            \hline
            \textbf{blank21} & blank & blank & blank\\
            \hline
            \textbf{blank69} & blank & blank & blank\\
            \hline
            \textbf{blank37} & blank & blank & blank\\
            \hline
        \end{tabular} 
    \end{table}
    
\subsection{Wnioski}

\section{Zadanie 2.}

\subsection{Opis problemu}


\subsection{Sposób rozwiązania}
\subsection{Wyniki}

\subsection{Wnioski}

\section{Zadanie 3.}

\subsection{Opis problemu}

\subsection{Sposób rozwiązania}

\subsection{Wyniki}

\subsection{Wnioski}

\section{Zadanie 4.}

\subsection{Opis problemu}
\subsection{Sposób rozwiązania}
\subsection{Wyniki}

\subsection{Wnioski}

\section{Zadanie 5.}

\subsection{Opis problemu}

\subsection{Sposób rozwiązania}

\subsection{Wyniki}

\subsection{Wnioski}

\section{Zadanie 6.}

\subsection{Opis problemu}

\subsection{Wyniki}

\subsection{Wnioski}

\section{Zadanie 7.}

\subsection{Opis problemu}

\subsection{Sposób rozwiązania}

\subsection{Wyniki}

\subsection{Wnioski}

\end{document}
